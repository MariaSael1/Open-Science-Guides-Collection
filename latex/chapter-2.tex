\documentclass{article}

\usepackage{caption}
                
\usepackage[backend=biber,hyperref=false,citestyle=authoryear,bibstyle=authoryear]{biblatex}
                
\bibliography{bibliography}
            
\usepackage{graphicx}
                
\usepackage{calc}
                
\newlength{\imgwidth}
                
\newcommand\scaledgraphics[2]{%
                
\settowidth{\imgwidth}{\includegraphics{#1}}%
                
\setlength{\imgwidth}{\minof{\imgwidth}{#2\textwidth}}%
                
\includegraphics[width=\imgwidth,height=\textheight,keepaspectratio]{#1}%
                
}
            
\begin{document}

\title{Open Science and Knowledge Justice}

\maketitle


\textbf{Contributors}: Kaan Ilgaz, Ümit Günes, My Linh Nguyen Thi, Lorenzo Vassao


\textbf{Tags / topics:} Knowledge Justice; equity; equity; justice


\subsection{Knowledge Justice: An Opportunity for Counter-expertise in Security vs. Science Debates}\label{H870315}


\begin{figure}
\scaledgraphics{086e7cc2-fafd-4cc2-8c8f-018ac3a9884d.png}{1}
\label{F24704971}
\end{figure}


\textbf{Guide name:} \emph{"Knowledge Justice: An Opportunity for Counter-expertise in Security vs. Science Debates"} \autocite{r_egert_knowledge_2017}


\textbf{Target group: }Researchers interested in Knowledge Justice and how important resources are unfairly distributed around the world.


\textbf{Type:} Journal


\textbf{Summary: }Knowledge Justice (Wissensgerechtigkeit) verbindet Prinzipien der sozialen Gerechtigkeit in wissenschaftliche Umgebungen. Dabei soll jeder die Möglichkeit haben, etwas dazu beitragen zu können oder Wissen zu erlangen. In letzter Zeit sieht man aber andersrum in den USA, wie versucht wird an Wissen über die H5N1 bzw. der Vogelgrippevirus zu gelangen. Dadurch löste sich eine Debatte zwischen Wissenschaftlern und Politikern über die Forschung. Zudem ist das Virus bereits ein Problem für Drittweltländer, die dieses Wissen sowieso nicht besitzen.


Das Konzept des Knowledge Justices zielt eine neue Denkweise über die Wissenschaft, wo alle betroffenen die nötige Expertise besitzen, um Probleme gemeinsam zu lösen.


\subsubsection{Knowledge Justice: Disrupting Library and Information Studies through Critical Race Theory}\label{H8244312}



\begin{center}
\begin{figure}
\scaledgraphics{dcc1d35b-4186-4b50-a5a6-cf54422a8064.jpeg}{0.5}
\label{F29320001}
\end{figure}


\end{center}


\textbf{Guide name:} \emph{Knowledge Justice: Disrupting Library and Information Studies through Critical Race Theory} \autocite{leung_knowledge_2021}


\textbf{Target group: }The target audience of this book are people who want to research basic information and the possibilities of Open Science. Furthermore, Open Science should be accessible to everyone and everyone should have equal rights.


\textbf{Type:} Recommendations for Practice


\textbf{Summary: }In Knowledge Justice, scholars from diverse ethnic backgrounds draw on critical race theory to challenge the fundamental principles, values and assumptions of library and information science in the United States. This is intended to lead the profession to understand how "white" supremacy affects practices, services, curricula, spaces, and policies.


The authors describe that a misconception of the neutrality and objectivity of library and information science comes from the influence of the different ethnicities of scholars. Through in-depth analyses of library and archival collections, scholarly communication, power hierarchies, epistemic domination, children's libraries, teaching and learning, digital humanities, and the education system, Knowledge Justice calls for the abolition of so-called "white supremacy" in order to create racial justice for every group of people.


\subsubsection{Open Science and Knowledge Justice: How It Started – How It’s Going?}\label{H480694}



\begin{center}
\begin{figure}
\scaledgraphics{e590694e-3a8f-4a2f-801f-704d7d8edbc0.png}{0.75}
\label{F3873651}
\end{figure}


\end{center}


\textbf{Guide name:} \emph{Open Science and Knowledge Justice: How It Started – How It’s Going?} \autocite{noauthor_open_2021}


\textbf{Target group: }People that want to research the Development of Open Science and Knowledge Justice


\textbf{Type:} Analysis


\textbf{Summary:} The article deals with the development of Knowledge Justice and Open Science. In recent decades, Open Science is said to have become increasingly relevant through many initiatives and other movements, and even Unseco (United Nations Educational, Scientific and Cultural Organization) has made recommendations in this regard. Open Science has changed the culture and this change should be promoted. GenR offers to help with these changes by working with the community.


\subsubsection{Open Science Promotes Diverse, Just, and Sustainable Research and Educational Outcomes}\label{H4807510}



\begin{center}
\begin{figure}
\scaledgraphics{23f9ee73-dda2-4a56-8083-4a618f9983b9.png}{0.75}
\label{F46402811}
\end{figure}


\end{center}


\textbf{Guide name}: \emph{Open Science Promotes Diverse, Just, and Sustainable Research and Educational Outcomes} \autocite{grahe_open_2019}


\textbf{Target group: }Researcher, who are interesed in Open Science which promotes diverse, just, and sustainable research and educational outcomes


\textbf{Type: }Analysis for evaluation purposes


\textbf{Summary:} Open science initiatives have become increasingly popular in recent decades. They offer the opportunity to promote diversity, equity and sustainability by supporting diverse, equitable and sustainable outcomes. This review examines models that demonstrate these aspects in the psychological economy and describes how open science initiatives promote these values. Diversity, equity and sustainability questions are offered that can be used to evaluate research outcomes.


\printbibliography[title={Bibliography}]
\end{document}
