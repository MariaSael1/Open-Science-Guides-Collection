\documentclass{article}

\usepackage{hyperref}
\usepackage{caption}
\usepackage{graphicx}
                
\usepackage{calc}
                
\newlength{\imgwidth}
                
\newcommand\scaledgraphics[2]{%
                
\settowidth{\imgwidth}{\includegraphics{#1}}%
                
\setlength{\imgwidth}{\minof{\imgwidth}{#2\textwidth}}%
                
\includegraphics[width=\imgwidth,height=\textheight,keepaspectratio]{#1}%
                
}
            
\begin{document}

\title{Readme}

\maketitle

\begin{figure}
\scaledgraphics{83db2204-7fbd-4954-bbf7-95dbd6ff4efa.png}{1}
\label{F89189351}
\end{figure}


\section{The Open Science Training Handbook}\label{H9514896}



A group of fourteen authors came together in February 2018 at the TIB (German National Library of Science and Technology) in Hannover to create an open, living handbook on Open Science training. High-quality trainings are fundamental when aiming at a cultural change towards the implementation of Open Science principles. Teaching resources provide great support for Open Science instructors and trainers. The Open Science training handbook will be a key resource and a first step towards developing Open Access and Open Science curricula and andragogies. Supporting and connecting an emerging Open Science community that wishes to pass on their knowledge as multipliers, the handbook will enrich training activities and unlock the community’s full potential.


Sharing their experience and skills of imparting Open Science principles, the authors (see \href{https://book.fosteropenscience.eu/en/#the-authors-and-the-book-sprint-facilitators}{below}) produced an open knowledge and educational resource oriented to practical teaching. The focus of the new handbook is not spreading the ideas of Open Science, but showing \textbf{how} to spread these ideas most effectively. The form of a book sprint as a collaborative writing process maximized creativity and innovation, and ensured the production of a valuable resource in just a few days.


Bringing together methods, techniques, and practices, the handbook aims at supporting educators of Open Science. The result is intended as a helpful guide on how to forward knowledge on Open Science principles to our networks, institutions, colleagues, and students. It will instruct and inspire trainers how to create high quality and engaging trainings. Addressing challenges and giving solutions, it will strengthen the community of Open Science trainers who are educating, informing, and inspiring themselves.


\section{Help us making the handbook better}\label{H3046972}



We welcome comments and feedback from everyone, irrespective of their expertise or background. The easiest way to do this is to leave a comment \href{https://book.fosteropenscience.eu/}{right here} by touching any paragraph with your mouse pointer and then clicking on the plus sign appearing next to that paragraph. If it is not working for you, you may consider using \href{https://via.hypothes.is/https://open-science-training-handbook.gitbook.io/book}{hypothes.is}. Also, you can create pull requests, either from within the Gitbook website or app, or with any tool you like. The handbook's content is maintained as \href{https://github.com/Open-Science-Training-Handbook}{this GitHub repository}.


\section{Let's run an Open Science training together}\label{H8885178}



Are you interested in running or attending trainings or webinars that make use of the Open Science Training Handbook? Get in touch with us at \href{mailto:elearning@fosteropenscience.eu}{elearning@fosteropenscience.eu} - we'd love to hear from you.

\begin{figure}
\scaledgraphics{1d13409d-cd97-40cd-ad54-a335b152b9af.png}{1}
\label{F61093731}
\end{figure}


\section{How to refer to the handbook}\label{H4508197}



Please consider citing the handbook when using the content. To cite the book, we recommend that you either refer to

\begin{itemize}
\item \href{https://book.fosteropenscience.eu/}{https://book.fosteropenscience.eu/}, which is the most friendly way to read the book (also available as \href{https://legacy.gitbook.com/download/pdf/book/open-science-training-handbook/book}{PDF}, \href{https://legacy.gitbook.com/download/epub/book/open-science-training-handbook/book}{ePub} and \href{https://legacy.gitbook.com/download/mobi/book/open-science-training-handbook/book}{Mobi}), to comment and to suggest changes, \emph{or}


\item \href{https://doi.org/10.5281/zenodo.1212496}{https://doi.org/10.5281/zenodo.1212496}, which is a citable DOI refering to a (hardly comprehensible) archived dump of the book.


\end{itemize}

\subsection{The Authors and the Book Sprint facilitators}\label{the-authors-and-the-book-sprint-facilitators}



Learn more about the authors and the book sprint facilitators, their experiences and inspiration, as well as their affiliation, contact information, Twitter and ORCID profiles, in the \href{https://book.fosteropenscience.eu/en/08AboutTheAuthorsAndFacilitators}{Handbook's last chapter}.


\subsection{Thank you to}\label{thank-you-to}


\begin{itemize}
\item Gwen Franck (EIFL, Belgium) for covering social media during the book sprint \& keeping us motivated with energizers


\item Patrick Hochstenbach (University of Gent, Belgium) for drawing the awesome cartoons and images


\item Vasso Kalaitzi (LIBER, Netherlands) for recording the really nice videos


\item Matteo Cancellieri (Open University, UK) for supporting us with all technical issues and creating the gitbook


\item Simon Worthington (TIB, Hannover, Germany) for providing advice with maintaining and converting bibliographic metadata


\end{itemize}

\subsection{Copyright statement}\label{copyright-statement}



The Open Science Training Handbook is an Open Educational Resource, and is therefore available under the \href{https://creativecommons.org/publicdomain/zero/1.0/}{Creative Commons Public Domain Dedication (CC0 1.0 Universal)}. You do not have to ask our permission to re-use and copy information from this handbook. Take note that some of the materials referenced in this book might be copyright protected — if so, this will be indicated in the text.


We have tried to acknowledge all our sources. If for some reason we have forgotten to provide you with proper credits, it has not been done with malicious intent. Feel free to contact us at \href{mailto:elearning@fosteropenscience.eu}{elearning@fosteropenscience.eu} for any corrections.


\subsection{Funding}\label{funding}



This project has received funding from the European Union’s Horizon 2020 research and innovation programme under grant agreement No. 741839.






\end{document}
