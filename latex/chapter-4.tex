\documentclass{article}

\usepackage{caption}
                
\usepackage[backend=biber,hyperref=false,citestyle=authoryear,bibstyle=authoryear]{biblatex}
                
\bibliography{bibliography}
            
\usepackage{graphicx}
                
\usepackage{calc}
                
\newlength{\imgwidth}
                
\newcommand\scaledgraphics[2]{%
                
\settowidth{\imgwidth}{\includegraphics{#1}}%
                
\setlength{\imgwidth}{\minof{\imgwidth}{#2\textwidth}}%
                
\includegraphics[width=\imgwidth,height=\textheight,keepaspectratio]{#1}%
                
}
            
\begin{document}

\title{Open Science and Data Science}

\maketitle


\textbf{Contributors:} Falkewitz, Philip;  Görzen, Linda; Matern, Johannes;  Shahbazi, Kian 


\textbf{Tags / topics:} Data Science; Machine Learning; Big Data; Python; Best practices; Reproducible research


\subsection{Big Data and Open Science Data}\label{H6205473}



\begin{center}
\begin{figure}
\scaledgraphics{25f071a6-9aec-4c08-bfd3-bfb548adfaee.jpg}{0.5}
\label{F77040021}
\end{figure}


\end{center}


 


\textbf{Guide name:} Big Data and Open Science Data \autocite{gutierrez_big_2015}


\textbf{Type:} step-by-step, instructions


\textbf{Target group:} Data researchers who are looking for a roadmap on Open Science Data


\textbf{Summary:} This article is about Big Data research and its content sharing. The goal here is to disclose benefits to support the promotion of Open Science. The article describes the problem of the high volume of Big Data and thus opacity, as well as the positive effect of data sharing of such data volumes. The article is divided into 4 sections.


1. problem definition


2. state of research


3. advantages of Open Science


4. future outlook


\section{Perspectives on open science and scientific data sharing}\label{H5120034}



\begin{center}
\begin{figure}
\scaledgraphics{6ab89e0c-e542-4afd-b586-eabe59773bc0.jpg}{0.5}
\label{F43484161}
\end{figure}


\end{center}





\textbf{Guide name:}  Perspectives on open science and scientific data sharing: an interdisciplinary workshop \autocite{destro_bisol_perspectives_2014}


\textbf{Type}: step-by-step, instructions, workshop


\textbf{Target group:} scholars and researchers of all scientific domains who are looking to share or access open data


\textbf{Summary:} This article is looking at Open Data and Open Science in general, and does not focus on a specific domain. The goal behind it is to promote communication and interaction between scholars who are working with (open access) papers. The content consists of summaries of presentations from a meeting. The article is structured in four issues:


1. The establishment of a common framework and a general discussion about principles of open data, values and opportunities.


2. Insights about scientific practices, especially how the open data movement is developing in specific scientific domains.


3. A case study of homan genomic, which was one of the first big shared documents, which demonstrated the boundaries between large scale data sharing, the boundaries of openness and protection of individual data.


4. A discussion about the public communication fo science and the role of the public in it. This point includes proposals for initiatives on open science. These are integrating a top-down initiative by the gevernments, institutions and journals, in combination with a bottom-up approch von the community. Popularizing the benefits is also a propsal, which is being made, which includes explaining the benefits.


\section{Support Your Data}\label{H2541051}



\begin{center}
\begin{figure}
\scaledgraphics{45bdf5de-7e5b-448d-820a-d3b3288a4dbe.png}{0.5}
\label{F88206441}
\end{figure}


\end{center}


\textbf{Guide name:} Support Your Data: A Research Data Management Guide for Researchers \autocite{borghi_support_2018}


\textbf{Type:} Toolset for self-assessment, series of short guides


\textbf{Target group:} researchers working in different institutional and disciplinary contexts


\textbf{Summary:} Researchers are faced with rapidly evolving expectations about how to manage and share their data, code and other research materials. To help them meet these expectations and generally manage and share their data more effectively, there are series of tools called Support Your Data.


These tools include a rubric designed to allow researchers to self-assess their current data management practices.


Included are self-assessments of their current data management practices and a series of short guides that provide actionable information on how to improve practices based on need or desire. These are designed to be easily adapted to the needs of researchers working in different institutional and disciplinary contexts.


\section{ Recommendations for open data science}\label{H2986141}





\begin{figure}
\scaledgraphics{10e3bc75-5c14-48e3-a161-d685105f455b.png}{1}
\label{F57414871}
\end{figure}





\textbf{Guide name:} „Recommendations for open data science“ \autocite{gymrek_recommendations_2016}


\textbf{Type:} instructions for action


\textbf{Target group:} Life sciences, but also most other scientific disciplines


\textbf{Summary: }The authors criticise that the computational analyses used in research are usually not published with the research results. This makes the research results non-transparent and difficult to understand. This practice needs to change in the sense of the open science movement. For this purpose, scientific communities should follow the guidelines presented:

\begin{enumerate}
\item \textbf{The tool software used should be made available or cited in public repositories.}


\item \textbf{Make pipelines available or cite them in public repositories}


\item \textbf{Teach data science to researchers}


\item \textbf{Publishers and reviewers must enforce reproducibility of computations}


\end{enumerate}

The authors refer to life science. However, the instructions for action can be applied to most other scientific disciplines.


\printbibliography[title={Bibliography}]
\end{document}
