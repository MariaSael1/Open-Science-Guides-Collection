\documentclass{article}

\usepackage{hyperref}
\usepackage{caption}
                
\usepackage[backend=biber,hyperref=false,citestyle=authoryear,bibstyle=authoryear]{biblatex}
                
\bibliography{bibliography}
            
\usepackage{graphicx}
                
\usepackage{calc}
                
\newlength{\imgwidth}
                
\newcommand\scaledgraphics[2]{%
                
\settowidth{\imgwidth}{\includegraphics{#1}}%
                
\setlength{\imgwidth}{\minof{\imgwidth}{#2\textwidth}}%
                
\includegraphics[width=\imgwidth,height=\textheight,keepaspectratio]{#1}%
                
}
            
\begin{document}

\title{Open Science and Knowledge Justice}

\maketitle


\textbf{Contributors}: Kaan Ilgaz, Ümit Günes, My Linh Nguyen Thi, Lorenzo Vassao


\textbf{Tags / topics:} Knowledge Justice; equality; equity; justice


\subsection{Equity vs. Equality: What’s the Difference?}\label{H7912011}



\begin{center}
\begin{figure}
\scaledgraphics{f0cce63a-bf54-49fe-bc74-0291d6668193.png}{0.75}
\caption*{Source: “Addressing Imbalance,” by Tony Ruth for the \href{https://designintech.report/2019/03/11/%F0%9F%93%B1design-in-tech-report-2019-section-6-addressing-imbalance/}{2019 Design in Tech Report.External link:}\emph{\href{https://designintech.report/2019/03/11/%F0%9F%93%B1design-in-tech-report-2019-section-6-addressing-imbalance/}{open\_}}}\label{F24704971}
\end{figure}


\end{center}


\textbf{Guide name: }Equity vs. Equality: What’s the Difference? \autocite{mphgw_the_george_washington_university_equity_2020}


\textbf{Target group: }public health masters students, health community, policy and system designers


\textbf{Type:} Guide for Masters course


\textbf{Summary: }The guide is an introduction to the difference between equality and equity. Firstly it explains the difference in terms of how society hands out resources and opportunities using the USA 1960 example \emph{\href{https://en.wikipedia.org/wiki/The_Giving_Tree}{The Giving Tree}} is an American children's picture book written and illustrated by Shel Silverstein. The core idea being that society is not a natural system and has inbuilt rewards that prioritise social groups.

\begin{quote}



\emph{Equity is a solution for addressing imbalanced social systems. Justice can take equity one step further by fixing the systems in a way that leads to long-term, sustainable, equitable access for generations to come.}


\end{quote}


The guide then shows example reccomendations from leading health institutions and how thes define the topic: World Health Organization (WHO), \href{https://www.who.int/healthsystems/topics/equity/en/}{equity is defined}:  U.S. Centers for Disease Control and Prevention (CDC) refers to \href{https://www.cdc.gov/minorityhealth/strategies2016/index.html}{health equity}: and, Or, as the \href{http://viablefuturescenter.org/racemattersinstitute/}{Race Matters Institute} desribes.


The guide then shows examples of how equality and equity differ in policy, for example: 

\begin{quote}



Equality - A city cuts the budget for 25 community centers by reducing the operational hours for all centers by the same amount at the same times. Equity - The city determines which times and how many hours communities actually need to use their community centers and reduces hours for centers that aren’t used as frequently.


\end{quote}


Short case studies of programmes are described and a list of additional resources given.


\subsection{Knowledge Justice: Disrupting Library and Information Studies through Critical Race Theory}\label{H8244312}



\begin{center}
\begin{figure}
\scaledgraphics{dcc1d35b-4186-4b50-a5a6-cf54422a8064.jpeg}{0.75}
\label{F29320001}
\end{figure}


\end{center}


\textbf{Guide name:} \emph{Knowledge Justice: Disrupting Library and Information Studies through Critical Race Theory} \autocite{leung_knowledge_2021}


\textbf{Target group: }The target audience of this book are people who want to research basic information and the possibilities of Open Science. Furthermore, Open Science should be accessible to everyone and everyone should have equal rights.


\textbf{Type:} Recommendations for Practice, LIS study


\textbf{Summary: }In Knowledge Justice, scholars from diverse ethnic backgrounds draw on critical race theory to challenge the fundamental principles, values and assumptions of library and information science in the United States. This is intended to lead the profession to understand how "white" supremacy affects practices, services, curricula, spaces, and policies.


The authors describe that a misconception of the neutrality and objectivity of library and information science comes from the influence of the different ethnicities of scholars. Through in-depth analyses of library and archival collections, scholarly communication, power hierarchies, epistemic domination, children's libraries, teaching and learning, digital humanities, and the education system, Knowledge Justice calls for the abolition of so-called "white supremacy" in order to create racial justice for every group of people.


\subsection{Labour of Love: An Open Access Manifesto for Freedom, Integrity, and Creativity in the Humanities and Interpretive Social Sciences}\label{H4374482}



\begin{center}
\begin{figure}
\scaledgraphics{e590694e-3a8f-4a2f-801f-704d7d8edbc0.png}{0.75}
\label{F3873651}
\end{figure}


\end{center}


\textbf{Guide name: }Labour of Love: An Open Access Manifesto for Freedom, Integrity, and Creativity in the Humanities and Interpretive Social Sciences\textbf{ }\autocite{pia_labour_2020}


\textbf{Target group: }Social sciences, academics, journal editors, provosts


\textbf{Type:} Manifesto, reccomendations


\textbf{Summary:} The article deals with the development of Knowledge Justice and Open Science. In recent decades, Open Science is said to have become increasingly relevant through many initiatives and other movements, and even Unseco (United Nations Educational, Scientific and Cultural Organization) has made recommendations in this regard. Open Science has changed the culture and this change should be promoted. GenR offers to help with these changes by working with the community.


\subsubsection{Open Science Promotes Diverse, Just, and Sustainable Research and Educational Outcomes}\label{H4807510}



\begin{center}
\begin{figure}
\scaledgraphics{23f9ee73-dda2-4a56-8083-4a618f9983b9.png}{0.75}
\label{F46402811}
\end{figure}


\end{center}


\textbf{Guide name}: \emph{Open Science Promotes Diverse, Just, and Sustainable Research and Educational Outcomes} \autocite{grahe_open_2019}


\textbf{Target group: }Researcher, who are interesed in Open Science which promotes diverse, just, and sustainable research and educational outcomes


\textbf{Type: }Analysis for evaluation purposes


\textbf{Summary:} Open science initiatives have become increasingly popular in recent decades. They offer the opportunity to promote diversity, equity and sustainability by supporting diverse, equitable and sustainable outcomes. This review examines models that demonstrate these aspects in the psychological economy and describes how open science initiatives promote these values. Diversity, equity and sustainability questions are offered that can be used to evaluate research outcomes.


\printbibliography[title={Bibliography}]
\end{document}
