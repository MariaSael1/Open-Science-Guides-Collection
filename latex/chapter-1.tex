\documentclass{article}

\usepackage{hyperref}
\begin{document}

\title{About the Book and Foreword}

\maketitle


\textbf{Tags / topics:} open science, introduction, motivation, guides


\subsection{Impressum}\label{H8810070}



Multi-format versions: \href{https://vivliostyle.vercel.app/#src=https://tibhannover.github.io/Open-Science-Guides-Collection/html/index.html&bookMode=true}{Webbook} | EPUB | PDF | source


Title: The Open Science Guides Collection


Place: Hannover, Germany


Date: May 2021


Edition / Version: Edition 1, Version 1.0


DOI: \href{https://doi.org/10.5281/zenodo.4740163}{10.5281/zenodo.4740163}


GitHub source: \href{https://github.com/TIBHannover/Open-Science-Guides-Collection}{https://github.com/TIBHannover/Open-Science-Guides-Collection} 


The publication is made from a larger collection of Open Science guides on a \emph{GenR} Zotero \href{https://www.zotero.org/groups/1838445/generation_r/collections/DND4FSHT}{collection}.


© 2021 the authors. The book and its components are licensed unter CC-BY 4.0 unless otherwise stated. \href{https://creativecommons.org/licenses/by-sa/4.0/}{https://creativecommons.org/licenses/by-sa/4.0/} 


Contributors: HsH. BIM-224, Open Knowledge, summer term 2021, Blümel. See GitHub \href{https://github.com/TIBHannover/Open-Science-Guides-Collection/blob/main/CONTRIBUTE.MD}{CONTRIBUTE.MD} 


\subsection{Foreword}\label{H592737}



Open Science has become an indispensable part of modern science. There are several definitions of "openness" in relation to different aspects of science - the \href{https://opendefinition.org/}{Open Definition} sets out principles as follows “Open means anyone can freely access, use, modify, and share for any purpose (subject, at most, to requirements that preserve provenance and openness).” \textbf{Practical guides} for the implementation of those principles in different areas such as research data or publishing are of great importance because they can be used right away. 


In this compendium, we compile important guides with their specific features and fields of application. The book was written as part of a student seminar at the Hanover University of Applied Sciences and Arts.

\end{document}
