\documentclass{article}

                
\usepackage{authblk}
                
\makeatletter
                
\let\@fnsymbol\@alph
                
\makeatother
            
\usepackage{caption}
                
\usepackage[backend=biber,hyperref=false,citestyle=authoryear,bibstyle=authoryear]{biblatex}
                
\bibliography{bibliography}
            
\usepackage{graphicx}
                
\usepackage{calc}
                
\newlength{\imgwidth}
                
\newcommand\scaledgraphics[2]{%
                
\settowidth{\imgwidth}{\includegraphics{#1}}%
                
\setlength{\imgwidth}{\minof{\imgwidth}{#2\textwidth}}%
                
\includegraphics[width=\imgwidth,height=\textheight,keepaspectratio]{#1}%
                
}
            
\begin{document}

\title{Open Science and Knowledge Justice}

\maketitle

\author{Kaan Ilgaz}
\author{Ümit Günes}
\author{My Linh Nguyen Thi}
\author{Lorenzo Vassao}
\affil{}





\textbf{Tags:} Knowledge Justice; Social Justice; equity; Gerechtigkeit


\subsubsection{Knowledge Justice: An Opportunity for Counter-expertise in Security vs. Science Debates}\label{H870315}


\begin{figure}
\scaledgraphics{086e7cc2-fafd-4cc2-8c8f-018ac3a9884d.png}{1}
\label{F24704971}
\end{figure}





\textbf{Guide name:} "Knowledge Justice: An Opportunity for Counter-expertise in Security vs. Science Debates" \autocite{r_egert_knowledge_2017}


\textbf{Type:} 


\textbf{Summary: }Knowledge Justice (Wissensgerechtigkeit) verbindet Prinzipien der sozialen Gerechtigkeit in wissenschaftliche Umgebungen. Dabei soll jeder die Möglichkeit haben, etwas dazu beitragen zu können oder Wissen zu erlangen. In letzter Zeit sieht man aber andersrum in den USA, wie versucht wird an Wissen über die H5N1 bzw. der Vogelgrippevirus zu gelangen. Dadurch löste sich eine Debatte zwischen Wissenschaftlern und Politikern über die Forschung. Zudem ist das Virus bereits ein Problem für Drittweltländer, die dieses Wissen sowieso nicht besitzen.


Das Konzept des Knowledge Justices zielt eine neue Denkweise über die Wissenschaft, wo alle betroffenen die nötige Expertise besitzen, um Probleme gemeinsam zu lösen.


\textbf{Autor}: Lorenzo Vassao


\subsubsection{Knowledge Justice: Disrupting Library and Information Studies through Critical Race Theory}\label{H8244312}



\begin{center}
\begin{figure}
\scaledgraphics{dcc1d35b-4186-4b50-a5a6-cf54422a8064.jpeg}{0.5}
\label{F29320001}
\end{figure}


\end{center}





\textbf{Guide name:} "Knowledge Justice: Disrupting Library and Information Studies through Critical Race Theory" \autocite{leung_knowledge_2021}


\textbf{Type:} Situationsbeschreibungen


\textbf{Summary:}

In Knowledge Justice beziehen sich die Wissenschaftler aus den verschiedenen Ethnien, auf die kritische Rassentheorie, um die grundlegenden Prinzipien, Werte und Annahmen der Bibliotheks- und Informationswissenschaft in den Vereinigten Staaten in Frage zu stellen. Dies soll den Berufsstand dazu zu bringen zu verstehen , wie die "weiße" Vorherrschaft Praktiken, Dienstleistungen, Lehrpläne, Räume und Richtlinien beeinflussen.


Die Autoren beschreiben, dass eine falsche Vorstellung der Neutralität und Objektivität der Bibliotheks- und Informationswissenschaft durch den Einfluss der verschiedenen Ethnien der Wissenschaftler zustande kommt. Durch tiefgreifende Analysen von Bibliotheks- und Archivsammlungen, wissenschaftlicher Kommunikation, Machthierarchien, epistemischer Vorherrschaft, Kinderbibliotheken, Lehren und Lernen, digitalen Geisteswissenschaften und dem Bildungssystem wird durch Knowledge Justice gefordert, die sogenannte "weiße Vorherrschaft" abzuschaffen, um die Rassengerechtigkeit für jede Menschengruppe zu erschaffen. 


\textbf{Autor}: Kaan Ilgaz


\subsubsection{Open Science and Knowledge Justice: How It Started – How It’s Going?}\label{H480694}


\begin{figure}
\scaledgraphics{e590694e-3a8f-4a2f-801f-704d7d8edbc0.png}{1}
\label{F3873651}
\end{figure}





\textbf{Guide name:} "Open Science and Knowledge Justice: How It Started – How It’s Going?" \autocite{noauthor_open_2021}


\textbf{Type:} Analysis


\textbf{Summary:} The article deals with the development of Knowledge Justice and Open Science. In recent decades, Open Science is said to have become increasingly relevant through many initiatives and other movements, and even Unseco (United Nations Educational, Scientific and Cultural Organization) has made recommendations in this regard. Open Science has changed the culture and this change should be promoted. GenR offers to help with these changes by working with the community.


\textbf{Autor}: Ümit Günes


\subsubsection{Open Science Promotes Diverse, Just, and Sustainable Research and Educational Outcomes}\label{H4807510}


\begin{figure}
\scaledgraphics{23f9ee73-dda2-4a56-8083-4a618f9983b9.png}{1}
\label{F46402811}
\end{figure}





\textbf{Guide name}: "Open Science Promotes Diverse, Just, and Sustainable Research and Educational Outcomes" \autocite{grahe_open_2019}


\textbf{Type: }Analysis for evaluation purposes


\textbf{Summary:} Open science initiatives have become increasingly popular in recent decades. They offer the opportunity to promote diversity, equity and sustainability by supporting diverse, equitable and sustainable outcomes. This review examines models that demonstrate these aspects in the psychological economy and describes how open science initiatives promote these values. Diversity, equity and sustainability questions are offered that can be used to evaluate research outcomes.


\textbf{Autor}: My Linh Nguyen Thi


\printbibliography[title={Bibliography}]
\end{document}
